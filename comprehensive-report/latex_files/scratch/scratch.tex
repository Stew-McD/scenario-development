Boosting recycling, re-use, repair, substitution, materials efficiency and circular design.
The recycling potential of electrolysers will only be realised after 2030, having a more tangible
effect towards 2040 \cite{worldbank2022hydrogen}. Recycling infrastructure for the collection, dismantling
and processing of the relevant products, components and materials needs to be put in place in
good time. R&D should be supported to develop innovative recycling methods offering high yield
rates and high-quality secondary materials. The fast uptake of electric vehicles in Europe is
accelerating the phase-out of conventional vehicles (with internal combustion engine) to cut
CO2 emissions by 2035. The platinum used in auto catalysts could therefore be an interesting
source of secondary raw materials for electrolyser manufacturing as early as 2030. Indeed,
closed-loop recycling of spent autocatalysts to recover materials such as platinum is a well-
established practice, and these flows could be channelled into the electrolyser industry. To be able
to confirm the secondary raw materials potential, the EU will need to develop recycling
infrastructure for platinum and iridium catalysts, develop and maintain data on secondary raw
materials relevant for electrolysers, and check material stocks and flows as well as competition
between sectors.

The time dimension is very important: new developments in mining and processing require time
to enter into operation, while recycling depends on the availability of sufficient end-of-life
volumes, and this may not happen before 2030 for some technologies.

The EU will not be completely self-reliant in its material needs, so it also needs to diversify its supply from
around the world, with trusted partners that mutually benefit from this cooperation  so that the EU`s supply
of critical raw materials becomes indeed sustainable. In parallel, the EU needs to improve its framework for
recycling and support innovation for materials efficiency and substitution. These are measures which the
European Commission is proposing with its European Critical Raw Materials Act.



Recycling, reuse and substitution. Enhancing the recycling of PGMs can reduce supply risks at
the refining stage. R&D efforts are needed to improve the efficiency of recycling technology and
the cost of recycling, creating a substantial economic driver for the expansion of recycling
activities. Alternatively, platinum could be substituted partially or totally with other metals such as
metal oxides, nitrides, carbide or chalcogenides. However, these options demonstrate poor stability
and low activity, making them difficult to substitute for platinum on an industrial scale.


Iridium can become a critical material for the deployment of electrolysers in the EU: its demand would exceed
the current global supply by 2050. It should be noted, however, that the demand could be relaxed if further
reduction in the catalysts loading is achieved from 2030 onwards, compared to the SRIA 2030 targets for PGMs.
In addition, primary materials demand can be further reduced by recycling and re-use practices: according to a
recent study \cite{worldbank2022hydrogen}, around 40\% of Iridium demand can be satisfied by such practices from 2030 onwards.

~\cite{jrc2023supplychain}