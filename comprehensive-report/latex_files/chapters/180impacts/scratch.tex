\cite{ayres1995lcacritique}
Life cycle analysis (LCA) is an increasingly important tool for environmental policy, and even for industry. Analysts are also interested in forecasting future materials/energy fluxes on regional and global scales, as a function of various economic growth and regulatory scenarios. A fundamental tenet of LCA is that every material product must eventually become a waste. To choose the ‘greener’ of two products or policies it is necessary to take into account its environmental impacts from ‘cradle to grave’. This includes not only indirect inputs to the production process, and associated wastes and emissions, but also the future (downstream) fate of a product. The first stage in the analysis is quantitative comparisons of materials flows and transformations. Energy fluxes are important insofar as they involve materials (e.g., fuels, combustion products). This can be an extremely valuable exercise, if done carefully. However, the data required to accomplish this first step are not normally available from published sources. Theoretical process descriptions from open sources may not correspond to actual practice. Moreover, so-called ‘confidential’ data are unverifiable (by definition) and may well be erroneous. In the absence of a formal materials balance accounting system, such errors may not be detected. A key thesis of this paper is that process data can, in many cases, be synthesized, using models based on the laws of thermodynamics and chemistry. While synthetic but possible data may not fully reflect the actual situation, it is far superior to ‘impossible’ data. Most of the recent literature on LCA focusses on the second stage of the analysis, namely the selection and evaluation of different, non-comparable environmental impacts (‘chalk vs. cheese’). This problem is, indeed, very difficult-and may well be impossible to solve convincingly -even at the conceptual level. However, the only approach that can make progress is one that utilizes monetization to the limits of its applicability, rather than one that seeks to by-pass (or ‘re-invent’) economics. Nevertheless, the evaluation problem is second in priority, for the simple reason that LCA has utility even if the evaluation technique is imperfect. On the other hand, LCA has no (or even negative) utility if the underlying physical data is wrong with respect to critical pollutants.




