\breaksection{Scenario development methods}\label{appendix:methods}

\autoref{tab:methods} provides an overview of the methods and tools considered, along with a brief description of each and its relevance to the specific context and objectives of the FutuRaM scenario development process.

\begin{landscape}
  \centering
  \small
  \begin{longtable}{|L{3cm}|L{7cm}|L{4cm}|L{5cm}|L{4cm}|}
      \caption{Scenario development methods}\label{tab:methods}\\
      \hline
      \rowcolor{headerblue} % Applying the header color
      \textbf{\textcolor{white}{METHOD}}& \textbf{\textcolor{white}{DESCRIPTION}}& \textbf{\textcolor{white}{KEY CHARACTERISTICS}}& \textbf{\textcolor{white}{LIMITATIONS}}& \textbf{\textcolor{white}{APPLICATION}}\\
      \hline
      \endfirsthead%
      \hline
      \multicolumn{5}{r}{\textcolor{headerblue}{\textit{{Continued on next page}}}} \\
      \endfoot%
      \rowcolor{white}
      \multicolumn{5}{c}{{\textcolor{headerblue}{\textit{\tablename\ \thetable{} --- Continued from previous page}}}} \\
      \hline
      \rowcolor{headerblue} % Applying the header color
      \textbf{\textcolor{white}{METHOD}}& \textbf{\textcolor{white}{DESCRIPTION}}& \textbf{\textcolor{white}{KEY CHARACTERISTICS}}& \textbf{\textcolor{white}{LIMITATIONS}}& \textbf{\textcolor{white}{APPLICATION}}\\
      \hline
      \endhead%
      \bottomrule
      \endlastfoot%
      \csvreader[late after last line=\ , separator=semicolon]{csvs/methods.csv}{}{
        \csvcoli& \csvcolii& \csvcoliii& \csvcoliv& \csvcolv\\
      }
    \end{longtable}
    
\end{landscape}