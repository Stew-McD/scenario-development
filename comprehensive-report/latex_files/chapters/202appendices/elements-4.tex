\breaksection{Drivers and factors for quantification}\label{appendix:elements-4}

The following \autoref{tab:elements-4} lists the categorised scenario elements that were quantified and incorporated into the modelling.

\begin{landscape}
  \centering
  \small
  \begin{longtable}{|C{1.5cm}|L{5.4cm}|C{1.8cm}|C{1.8cm}|C{1.8cm}|C{1cm}|C{1cm}|C{1cm}|C{6cm}|}
  \caption[List of scenario elements categorised for quantification]{List of scenario elements categorised for quantification. The roman numerals in the columns BAU, REC, and CIR represent the magnitude of the future trend for the element in the scenario. Internal, external, and outside refer to the classification type of the scenario element}\label{tab:elements-4}\\
        \hline
        \rowcolor{headerblue} % Applying the header color
        \textcolor{white}{\textbf{DOMAIN}} & \textcolor{white}{\textbf{ELEMENT}} & \textcolor{white}{\textbf{INTERNAL}} & \textcolor{white}{\textbf{EXTERNAL}} & \textcolor{white}{\textbf{OUTSIDE}} & \textcolor{white}{\textbf{BAU}} & \textcolor{white}{\textbf{REC}} & \textcolor{white}{\textbf{CIR}} & \textcolor{white}{\textbf{PARAMETERS AFFECTED}} \\
        \hline
        \endfirsthead%
        \hline
        \multicolumn{9}{r}{\textcolor{headerblue}{\textit{{Continued on next page}}}} \\
        \endfoot%
        \rowcolor{white}
        \multicolumn{9}{c}{{\textcolor{headerblue}{\textit{\tablename\ \thetable{} --- Continued from previous page}}}} \\
        \hline
        \rowcolor{headerblue} % Applying the header color       
        \textcolor{white}{\textbf{DOMAIN}} & \textcolor{white}{\textbf{ELEMENT}} & \textcolor{white}{\textbf{INTERNAL}} & \textcolor{white}{\textbf{EXTERNAL}} & \textcolor{white}{\textbf{OUTSIDE}} & \textcolor{white}{\textbf{BAU}} & \textcolor{white}{\textbf{REC}} & \textcolor{white}{\textbf{CIR}} & \textcolor{white}{\textbf{PARAMETERS AFFECTED}} \\ \hline
        \endhead%
        \bottomrule
        \endlastfoot%
    \csvreader[late after last line=\ , separator=semicolon]{csvs/elements-4.csv}{}{
      \csvcoli& \csvcolii& \csvcoliii& \csvcoliv& \csvcolv& \csvcolvi &\csvcolvii &\csvcolviii &\csvcolix \\
    } 
  \end{longtable}
\end{landscape}