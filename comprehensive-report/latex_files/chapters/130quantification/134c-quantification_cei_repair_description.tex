\subsection{Repair: \textit{Description}}~\cite{cordella2021repairlca, wieser2018,svensson2018repair, hernandez2020repair, eu2019repair}

\subsubsection{Definition}
Right to repair refers to the concept that end users, business users as well as
consumers, of (generally) technical, electronic or automotive devices should be
allowed to freely repair these products. Four requirements are of particular
importance:

\begin{itemize}
    \item The device should be constructed and designed in a manner that allows repairs
          to be made easily;
    \item End users and independent repair providers should be able to access the original
          spare parts and necessary tools (software as well as physical tools) at fair
          market conditions;
    \item Repairs should, by design, be possible and not be hindered by software
          programming; and
    \item The repairability of a device should be clearly communicated by the
          manufacturer.
\end{itemize}


\boxparameter{Repair}{Stable}{Stable}{Strong increase}


\subsubsection{Context}

Discarded products are often viable goods that can be repaired but are often
tossed prematurely, resulting in 35 million tons of waste, 30 million tons of
resources and 261 million tons of greenhouse gas emissions in the EU every
year~\cite{moeslinger2022repair}

Repairing is one of the most relevant strategies within the Circular Economy
(CE) concept since it contributes to waste prevention and extends product and
components' lifespan. Thus, reparability becomes an essential issue from the
early product design phases, where materials, geometries, and joints are
defined. Despite some repairability indicators that can be found in the
literature and are applied worldwide, there is a lack of connection between
repairability and the early decision-making process for improving it from the
design of components or subsystems of a product.

However, repair is often seen as difficult by consumers. The `right to repair'
initiative complements several other proposals presented by the Commission to
achieve sustainable consumption throughout the entire lifecycle of a product,
setting the framework for a true `right to repair' across the EU. Obstacles to
owner repair can lead to higher consumer costs or drive consumers to single-use
devices instead of making repairs.

The right to repair is a legal right for owners of devices and equipment to
freely modify and repair products such as automobiles, electronics, and farm
equipment. This right is framed in opposition to restrictions such as
requirements to use only the manufacturer's maintenance services, restrictions
on access to tools and components, and software barriers.

A right to repair can exist either in a closed access system, where the
consumer is restricted to the repair services provided by the manufacturer or
authorized repairers --- a situation closer to the current reality. Or, a right
to repair can evolve in an open access system, which implies full access to
spare parts, tools, repair manuals and digital permission to repair. Policy
options for a right to repair differ based on whether they encourage one or the
other approach. Some argue an open access system is the only form of right to
repair that is consumer-empowering and can yield the expected benefits. Others
argue for a more complex system, moving towards open access but with some
safeguards on a sectoral or product category basis. A cost-benefit analysis
could help identify the sectors or product categories where a full open-access
system would be most beneficial.

The goals of the right to repair are to favour repair instead of replacement and
make such repairs more affordable leading to a more sustainable economy and
reduction in waste.

\subsubsection{International and European Right to Repair Initiatives}~\cite{svensson2018repair, hernandez2020repair, eu2019repair}
\begin{itemize}
    \item \textbf{Availability of Spare Parts and Repair Information:}
          \begin{itemize}
              \item US state-level legislation includes laws like Massachusetts' requirement for
                    car manufacturers to provide repair tools and information.
              \item The EU has measures like France's mandate for sellers to inform about the
                    availability of spare parts, and Slovenia's requirement for maintenance and
                    spare parts for at least 3 years after guarantee expiration.
          \end{itemize}

    \item \textbf{Legal Guarantees:}
          \begin{itemize}
              \item European legal guarantee periods often exceed the EU directive's minimum,
                    encouraging repair culture.
              \item For example, Sweden has a 3-year guarantee period, and Finland ties the period
                    to the expected lifespan of the product.
          \end{itemize}

    \item \textbf{Design Requirements:}
          \begin{itemize}
              \item Legislation like Washington State's (USA) proposed fair repair bill is aimed
                    at promoting repairable product designs by prohibiting the creation of
                    electronics that obstruct repairability.
          \end{itemize}

    \item \textbf{Financial Incentives:}
          \begin{itemize}
              \item Cities like Graz offer subsidies for electronic device repairs and countries
                    like Belgium provide écochèques to incentivize repair over replacement.
          \end{itemize}

    \item \textbf{Copyright Law Exemptions:}
          \begin{itemize}
              \item In the US, certain copyright law exemptions facilitate repairability, such as
                    the ability to unlock phones, although the exemption renewal process is
                    cumbersome.
          \end{itemize}

    \item \textbf{Consumer Information:}
          \begin{itemize}
              \item France's reparability index helps inform consumers by rating products on
                    repairability criteria, promoting repair-friendly designs.
          \end{itemize}

    \item \textbf{Voluntary Labels and Green Public Procurement:}
          \begin{itemize}
              \item Ecolabels such as EPEAT and various national labels incorporate repairability
                    to different degrees.
              \item Green Public Procurement practices push the market towards sustainable,
                    repairable products.
          \end{itemize}

    \item \textbf{Communication and Awareness:}
          \begin{itemize}
              \item Initiatives include repair-focused websites, awareness campaigns, and the
                    establishment of repair hubs to build a repair-oriented culture.
          \end{itemize}
\end{itemize}

\subsubsection{Implementation in EU Law}~\cite{eu2023repair, moeslinger2022repair}

European Product Policy has to date focused on the environmental performance of
products via the Ecodesign and Ecolabelling Directives. The Ecodesign Directive
sets minimum standards of performance for products, which results in poorly
performing products being removed from the market whilst also driving
innovation in the design and manufacture of new products to improve their
performance. The Ecolabelling Directive provides consumers with clear
information on product performance to inform their buying decisions. Originally
cast for energy-using products, the directives have been extended to
energy-related products and the assessment methodologies have been developed to
include other aspects including materials and water consumption.

Further measures considered include:

\begin{itemize}
    \item Amending Directive 2005/29/EC to prohibit presenting products as allowing
          repair when such repair is not possible, as well as omitting to inform
          consumers that it is not possible to repair goods in accordance with legal
          requirements.
    \item Amending Directive 2005/29/EC to prohibit omitting to inform the consumer that
          the good is designed to limit its functionality when using consumables, spare
          parts, or accessories that are not provided by the original producer.
    \item Traders to provide, before the conclusion of the contract, for all types of
          goods, where applicable, the reparability score of the good as provided by the
          producer in accordance with Union law, to allow consumers to make an informed
          transactional decision and choose goods that are easier to repair.
    \item Ensuring information such as on the availability of spare parts and a repair
          manual, should no reparability score be available at the Union level.
\end{itemize}

To this end, new ‘Digital Product Passports’ providing information about
products’ environmental sustainability, will empower consumers and businesses
to make informed choices when purchasing products, facilitate repairs and
recycling, and improve transparency about products’ lifecycle impacts on the
environment. The passports also help public authorities to better perform
checks and controls.

In addition, as part of the implementation of the EU Circular Economy Action
Plan~\cite{eu2020circ}, the European Commission has carried out a study for the
analysis and development of a possible scoring system to inform about the
ability to repair and upgrade products~\cite{eu2019repair} and has an ongoing
project in the Product Bureau to develop and propose new
metrics~\cite{eu2023repairproject, eu2022repair}.

\subsubsection{Development of a metric for repairability}~\cite{eu2019repair, moeslinger2022repair, ruizpastor2023repair, eu2023repair, barros2023repair,eu2022repair, eu2023repairproject,sagar2022repair}

The trend in consumer goods towards reduced durability and repairability has
been contributing to an increase in waste electronic and electrical equipment
(WEEE). The Organization for Economic Co-operation and Development has
suggested that extending product lifetimes through enhanced durability and
repairability is a viable solution to this growing issue. The European
Commission's Circular Economy Action Plan reinforces this viewpoint, advocating
for maintaining the value in products for as long as possible by imposing
durability and repairability requirements. In response, several scoring systems
for repairability have been developed to guide standardization efforts, aid
market surveillance authorities, and inform consumer decision-making.

For a scoring system to be effective, it should provide an objective evaluation
of repairability that aligns with the established design principles in the
literature. Comparative analyses of various repairability scoring systems for
different products have been undertaken in previous studies. However, the
thoroughness of these systems is sometimes not fully evaluated, and some of the
most recent systems have not been comprehensively reviewed.

Literature on the subject identifies specific design features and principles
that significantly affect product repairability, and these should be central to
any scoring system aimed at accurately measuring repairability. Assessing these
design elements against selected scoring systems can shed light on their
inclusiveness.

The objectivity of these scoring systems is another critical aspect, evaluated
by examining their scoring methodologies. Selection criteria for these systems
include their availability in English, the use of quantitative or
semi-quantitative assessment methods to enable objective comparison, and their
recognition as the most current versions from their respective issuing
organizations or groups.

In 2021, France took a pioneering step by integrating the reparability index
into national legislation.~\cite{france2020repair} This move compels producers
to transparently communicate the repairability of their products through
consumer labelling. The reparability index stands as a key development in
empowering consumers to make informed choices regarding the repairability of
products. The widespread issue of repairing common electronic devices like
laptops and smartphones often stems from the unavailability of tools, spare
parts, or repair instructions.

An exemplary repair index would encompass elements such as product design, the
availability of repair information, and additional services like the
availability of spare parts. These aspects are crucial for the repair process.
Data indicates that a substantial number of electronic product repairs are
hindered by the lack of available spare parts.

France's method mandates transparency regarding product repairability, yet
relies on producers' self-assessment, prompting questions about the objectivity
of such evaluations. The rapid implementation is advantageous, but the
credibility of self-assessment remains a concern.

With sustainability becoming increasingly important, France's repairability
index marks an assertive step towards the broader adoption of such measures.
Looking ahead, enhancements like a durability index may offer greater insights
into the long-term usability of products.

In parallel, organizations such as TÜV SÜD are actively supporting the
repairability testing landscape, aligning with standards like the French
Repairability Scoring Index to ensure products fulfil specified repairability
criteria~\cite{tuv2023repair}. Their approach factors in documentation,
disassembly, and the availability of spare parts and repair services,
highlighting a practical, though less detailed, framework compared to France's
comprehensive index.

\subsubsection{Benefits and risks}

\subsubsubsection{Environmental Benefits and Risks}~\cite{eu2019repair, boulos2015durability, }

The implementation of the right to repair holds considerable promise for the
reduction of environmental impacts if applied appropriately. It must be
recognized that electronic equipment replacement often occurs not solely due to
product failure. Influencing factors such as perceived obsolescence contribute
significantly, as evidenced by a study in Austria revealing only 30\% of
replacements were attributable to malfunctioning products~\cite{wieser2018}.

Direct measurement of the impact of a right to repair is challenging, with the
need to consider additional variables such as obsolescence perception, device
performance, and consumer behaviour trends in determining potential extensions
in consumer electronics' average lifespan.

Moreover, the environmental benefit of repair is contingent not only on the
increased product lifespan post-repair but also on the environmental footprint
of the spare parts required for repair. Circuit boards, for example, carry
substantial environmental impacts, and their replacement could still result in
significant environmental costs. Common repairs typically involve less
impactful components such as screens, casings, batteries, or
software~\cite{cordella2021repairlca}.

Cordella et al.~\cite{cordella2021repairlca} report that compared to the
baseline of replacing smartphones every two years,
extending the device's life through repair can substantially diminish the
carbon footprint. A one-year extension, with a battery change, can reduce
greenhouse gas (GHG) emissions by 29\%, and by 44\% with a two-year extension.

With 472 million Europeans owning a mobile phone, there are \(8.11 \text{ Mt
    CO}_2\text{-eq.}\) in annual emissions solely from phones. Extending the life
of a mobile phone by just one year, including component replacements, could
reduce emissions to \(6.23 \text{ Mt CO}_2\text{-eq.}\)
annually~\cite{moeslinger2022repair}. A further extension by an additional year
could decrease emissions to \(4.91 \text{ Mt CO}_2\text{-eq.}\), effectively
removing the equivalent of over 2 million cars from European roads.

Nevertheless, these potential reductions should be interpreted with caution as
they are based on estimations and may not account for potential rebound
effects. For instance, economic savings from prolonged use of electronic
devices could lead to rebound effects where savings are offset by additional
consumption stemming from the economic savings~\cite{makov2018sharing}.

Finally, repair activities offer a more energy-efficient alternative within the
Circular Economy compared to recycling and remanufacturing, which demand
extensive energy input and high material throughput. When feasible, repair
should be prioritized over other circular economy
strategies~\cite{eu2019repair}.

\subsubsubsection{Manufacturers' Perspective}
\begin{itemize}
    \item Compliance with eco-design standards could reduce profit margins.
    \item Risk of increased liability and the need to ensure long-term availability of
          spare parts.
    \item Potential decrease in turnover due to extended product lifecycles.
    \item Reduction in EU imports could foster the EU's technological independence, as per
          the EU Chips Act.
    \item Loss in turnover potentially offset by repair services and spare parts supply.
\end{itemize}

\subsubsubsection{Broader Economic and Environmental Implications}
\begin{itemize}
    \item Right to Repair could enhance competitiveness by increasing product longevity
          and added value.
    \item Positive impact on professional repair services, spare parts provision, and
          tool providers.
    \item SMEs and local repair shops likely to benefit significantly.
    \item Potential for the development of new European leaders in repair services.
    \item A more repairable design could improve recycling processes and increase
          component harvesting.
\end{itemize}

\subsubsection{Relevance of Repairability to Critical Raw Materials in Waste Streams}

The integration of the `Right to Repair' ethos and the promotion of
repairability has implications for the management of Critical Raw Materials
(CRMs) across various waste streams, such as BATT (waste batteries), ELV
(end-of-life vehicles), WEEE (waste electrical and electronic equipment), and
CDW (construction and demolition waste).

\subsubsubsection{Batteries (BATT)}
Batteries are a crucial repository of CRMs like lithium, cobalt, and nickel. Enhancing their repairability can lead to:
\begin{itemize}
    \item Refurbishing batteries for second-life applications.
    \item Design modifications for easier replacement of battery cells.
    \item Reduced extraction of new raw materials, mitigating the environmental
          footprint.
\end{itemize}

\subsubsubsection{End-of-Life Vehicles (ELV)}
Vehicles are a significant source of CRMs such as platinum and palladium (catalytic converters) and rare earth elements (electronics and magnets). `Right to Repair' can:
\begin{itemize}
    \item Influence design changes for modularity and ease of part replacement.
    \item Prolong the utility of CRMs and lessen new resource extraction.
\end{itemize}

\subsubsubsection{Waste Electrical and Electronic Equipment (WEEE)}
The WEEE stream contains valuable CRMs like gold, silver, and rare earth elements. Promoting repairability results in:
\begin{itemize}
    \item Prolonged life spans for electronic devices.
    \item A reduction in the volume of CRMs entering the waste stream.
    \item Conservation of valuable materials through repair and refurbishment.
\end{itemize}

\subsubsubsection{Construction and Demolition Waste (CDW)}
CRMs feature in many building materials as well as wind turbines which are part of this waste stream, and advocating for repairability in construction can:
\begin{itemize}
    \item Lead to buildings designed for deconstruction, not demolition.
\end{itemize}

The emphasis on repairability and `Right to Repair' legislation can lead to
reduced CRM demand, decreased environmental impact through less mining,
creation of economic incentives for repair industries, and improved resource
security by minimizing reliance on raw material extraction. This approach is in
line with fostering a circular economy, aiming for a sustainable management of
resources within the EU.


\sectionEndlines
\clearpage