\subsection{Future product and waste composition: \textit{Description}}

Future compositions, technologies and products will be assessed based on technology outlooks and stakeholder
interviews and may include sector-specific Delphi surveys. Information needs and availability for composition data
as well as the type of relevant recoverable embodied SRMs varies across the waste streams. Thus, specific data
collection strategies will be developed and used for each waste stream.

\boxblue{Task 2.2 and 2.3}{Following the scenarios from T2.1, the material compositions and future products for each sector will be determined based on the product and commodity demand and technology realisation (T2.2). This task will be coupled to the data collection in WP3 and WP4}

\subsubsection{Definition}


\boxparameter{Future product and waste composition}{Stable}{Strong shift toward design for design for recycling, remanufacturing, recovery}{Strong shift toward durability and design for re-X, especially re-use, repair}


\subsubsection{Method}

The general method for determining future product and waste composition will be based on the following steps:

\begin{itemize}
    \item \textbf{Step 1:} Collection of data on current product and waste composition (WP3)
    \item \textbf{Step 2:} Grouping of code lists into broader categories (WP3)
    \item \textbf{Step 3:} Identification of future products based on technology outlooks, literature review, and stakeholder interviews
    \item \textbf{Step 4:} Estimation of constraints for market entry, replacement, and market share of future products (WP2) in each scenario
    \item \textbf{Step 5:} Modelling of future product and waste composition based on the results of steps 1-4
\end{itemize}

% \subsubsection{Context}



% \subsubsection{International and European Trends}

% \subsubsection{Implementation in EU Law}


% \subsubsection{Development of a metric for XXX}

% \subsubsection{Benefits and risks}

% \subsubsubsection{Environmental Benefits and Risks}

% \subsubsubsection{Manufacturers' Perspective}


% \subsubsubsection{Broader Economic and Environmental Implications}


\subsubsection{Relevance of future product and waste composition to Critical Raw Materials in Waste Streams}

The integration of future product and waste composition has implications for the management of Critical Raw Materials
(CRMs) across various waste streams, such as BATT (waste batteries), ELV
(end-of-life vehicles), WEEE (waste electrical and electronic equipment), and
CDW (construction and demolition waste).

\subsubsubsection{Batteries (BATT)}
\begin{itemize}
    \item 
\end{itemize}

\subsubsubsection{End-of-Life Vehicles (ELV)}
\begin{itemize}
    \item 
\end{itemize}

\subsubsubsection{Waste Electrical and Electronic Equipment (WEEE)}
\begin{itemize}
    \item 
\end{itemize}

\subsubsubsection{Construction and Demolition Waste (CDW)}

\begin{itemize}
    \item
\end{itemize}

\sectionEndlines
\clearpage
