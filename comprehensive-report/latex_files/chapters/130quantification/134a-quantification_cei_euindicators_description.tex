
\subsection{The EU Circular Economy Indicators: \textit{Description}}

\textbf{Main sources:}~\cite{eu2018cemonitoring,eurostat2023data,eu2023cemonitoring}


\boxparameter{The EU Circular Economy Indicators}{Stable}{Strong increase in recycling and recovery related indicators}{Strong in circularity related indicators}

\subsubsection{Economic Indicators}

\subsubsubsection{CEI\_CIE011:\\ Persons employed in circular economy sectors \& \\ CEI\_CIE012:\\ Private investment and gross added value related to circular economy sectors} 
\textbf{Indicator metadata:}   \href{https://ec.europa.eu/eurostat/cache/metadata/en/cei_cie011_esmsip2.htm}{\faExternalLink}

{\large\textbf{Context:}}  \\ \indent
Targets economic activities that contribute to the circular economy, delineating those activities through established environmental policy frameworks and classifications.

{\large\textbf{Indicator Description:}}  \\ \indent
The indicator encompasses ``Private investments'', ``Persons employed'' and ``Gross value added''. Eurostat has developed a method to derive these key economic variables, incorporating a multi-step approach: establishing a conceptual framework based on international environmental policy definitions, mapping and classifying relevant activities against an integrated system of economic classifications (using NACE, CPA, and PRODCOM codes), and finally compiling data using defined estimation procedures. The primary outputs of this process are the measurements of employment in FTE, gross value added at factor cost, and investments in tangible goods, each quantified in million euros.

{\large\textbf{Unit:}}  \\ \indent
Economic metrics are presented in million euros, with employment figures given in full-time equivalents (FTE); both sets of figures are also contextualised as percentages of GDP and total employment, respectively.

{\large\textbf{Source Data:}}  \\ \indent
Data is sourced from a combination of Structural Business Statistics, National Accounts, Prodcom surveys, and the Labour Force Survey, enriched by additional sector-specific statistics.

\subsubsubsection{CEI\_CIE020: Patents related to recycling and secondary raw materials}
\textbf{Indicator metadata:}   \href{https://ec.europa.eu/eurostat/cache/metadata/en/cei_cie020_esmsip2.htm}{\faExternalLink}

{\large\textbf{Context:}}  \\ \indent
This indicator is integral to the Circular Economy set, focusing on `competitiveness and innovation' and serving to gauge progress towards a more circular economy.

{\large\textbf{Indicator Description:}}  \\ \indent
The indicator enumerates the number of patent families pertinent to recycling and secondary raw materials, leveraging the Cooperative Patent Classification to ensure unique counts.

{\large\textbf{Unit:}}  \\ \indent
The unit of measure is the number of patent families, with a secondary metric of patents per million inhabitants.

{\large\textbf{Source Data:}}  \\ \indent
Sourced from the European Patent Office (EPO), the data are extracted and analyzed by the Joint Research Centre (JRC), using the PATSTAT database.

\subsubsubsection{CEI\_PC030: Resource productivity}
\textbf{Indicator metadata:}   \href{https://ec.europa.eu/eurostat/cache/metadata/en/cei_pc030_esmsip2.htm}{\faExternalLink}

{\large\textbf{Context:}} \\
Embedded within the Circular Economy indicator suite, this metric tracks progress in `Production and consumption', emphasizing material use efficiency to gauge economic growth relative to resource use.

{\large\textbf{Indicator Description:}}  \\ \indent
Resource productivity is articulated as GDP over DMC, showcasing the efficiency of material utilization within an economy. This indicator assists in understanding the dynamics between economic performance and environmental pressure.

{\large\textbf{Unit:}}  \\ \indent
Measured in three distinct units: euro per kg in chain-linked volumes (2015), PPS per kg, and as an index (2000=100) for temporal and spatial comparisons.

{\large\textbf{Source Data:}}  \\ \indent
The European Statistical System (ESS) supplies the data, with Eurostat disseminating information on DMC and GDP, derived from the Material Flow Accounts and GDP and main components datasets, respectively.

\subsubsection{Waste and Material Indicators}

\subsubsubsection{CEI\_PC020: Material footprint}
\textbf{Indicator metadata:}   \href{https://ec.europa.eu/eurostat/cache/metadata/en/cei_pc020_esmsip2.htm}{\faExternalLink}

{\large\textbf{Context:}}  \\ \indent
The `Material footprint' indicator is a critical component of the Circular Economy monitoring framework, highlighting the `production and consumption' thematic area. It reflects the EU's impact on global resources, pertinent to the EU's consumption exceeding its production, especially concerning goods manufactured in Asia and consumed in Europe.

{\large\textbf{Indicator Description:}}  \\ \indent
This indicator assesses the global demand for material extraction driven by EU consumption and investment. The Material Footprint provides insight into the environmental burden shifted to other regions due to the EU`s consumption patterns. It is expressed through the Raw Material Consumption (RMC) metric, indicating the material extraction required for goods consumed within the EU.

{\large\textbf{Unit:}}  \\ \indent
The unit of measure is tonnes per capita.

{\large\textbf{Source Data:}}  \\ \indent
Data source: European Statistical System (ESS)
Data provider: Statistical Office of the European Union (Eurostat).
Material flow accounts in raw material equivalents -- modelling estimates (env\_ac\_rme).~\href{https://ec.europa.eu/eurostat/web/products-datasets/-/env_ac_rmefd}{\faExternalLink}
Material flow accounts in raw material equivalents by final uses of products - modelling estimates (env\_ac\_rmefd).~\href{https://ec.europa.eu/eurostat/web/products-datasets/-/env_ac_rmefd}{\faExternalLink}

\subsubsubsection{CEI\_PC031: Generation of municipal waste per capita}
\textbf{Indicator metadata:}   \href{https://ec.europa.eu/eurostat/cache/metadata/en/cei_pc031_esmsip2.htm}{\faExternalLink}

{\large\textbf{Context:}}  \\ \indent
The `Generation of municipal waste per capita' indicator is integral to the Circular Economy indicator set, falling under the `production and consumption' thematic area. It underscores efforts to sustain product and material value in the economy, minimize waste generation, and drive waste prevention strategies in alignment with the Waste Hierarchy.

{\large\textbf{Indicator Description:}}  \\ \indent
This indicator tracks municipal waste generated and managed by local authorities or entities appointed by them. It predominantly accounts for household waste, although it may include waste from commercial activities, offices, and public institutions, reflecting consumer behaviour and the impact of waste reduction measures.

{\large\textbf{Unit:}}  \\ \indent
The unit of measure is kilograms per capita, based on the annual average population.

{\large\textbf{Source Data:}}  \\ \indent
The data is provided by Eurostat, consistent with the high-quality standards of the European Statistical System (ESS), deriving from the Municipal waste by waste operations report, collected under the OECD/Eurostat Joint Questionnaire. Data submission is voluntary, known as a `gentlemen's agreement'.

\subsubsubsection{CEI\_PC032: \\Generation of waste excluding major mineral wastes per GDP unit}
\textbf{Indicator metadata:}   \href{https://ec.europa.eu/eurostat/cache/metadata/en/cei_pc032_esmsip2.htm}{\faExternalLink}

{\large\textbf{Context:}}  \\ \indent
The `Generation of municipal waste per capita' indicator is integral to the Circular Economy indicator set, falling under the `production and consumption' thematic area. It underscores efforts to sustain product and material value in the economy, minimize waste generation, and drive waste prevention strategies in alignment with the Waste Hierarchy.

{\large\textbf{Indicator Description:}}  \\ \indent
This indicator tracks municipal waste generated and managed by local authorities or entities appointed by them. It predominantly accounts for household waste, although it may include waste from commercial activities, offices, and public institutions, reflecting consumer behaviour and the impact of waste reduction measures.

{\large\textbf{Unit:}}  \\ \indent
The unit of measure is kilograms per capita, based on the annual average population.

{\large\textbf{Source Data:}}  \\ \indent
The data is provided by Eurostat, consistent with the high-quality standards of the European Statistical System (ESS), deriving from the Municipal waste by waste operations report, collected under the OECD/Eurostat Joint Questionnaire. Data submission is voluntary, known as a gentlemen's agreement.

\subsubsubsection{CEI\_PC034: Waste generation per capita}
\textbf{Indicator metadata:}   \href{https://ec.europa.eu/eurostat/cache/metadata/en/cei_pc034_esmsip2.htm}{\faExternalLink}

{\large\textbf{Context:}}  \\ \indent
The `Waste generation per capita' indicator is a key component of the Circular Economy monitoring framework, aimed at assessing the effectiveness of EU policies focused on waste reduction and resource efficiency within the `production and consumption' thematic area.

{\large\textbf{Indicator Description:}}  \\ \indent
This indicator reflects the total waste generation within a country, including major mineral wastes from all economic activities and households. It is an essential measure for evaluating the impact of waste prevention measures, allowing comparison of Member States' performance over time.

{\large\textbf{Unit:}}  \\ \indent
The unit of measure is kilogram per capita

{\large\textbf{Source Data:}}  \\ \indent
The data originates from the European Statistical System (ESS), specifically Eurostat, which collates information reported by countries under the Waste Statistics Regulation (EC) No 2150/2002.

\subsubsubsection{CEI\_SRM030: Circular material use rate}
\textbf{Indicator metadata:}   \href{https://ec.europa.eu/eurostat/cache/metadata/en/cei_srm030_esmsip2.htm}{\faExternalLink}

{\large\textbf{Context:}}  \\ \indent
As a core metric within the Circular Economy indicator set, the `Circular material use rate' is crucial for monitoring advancements in the utilization of `secondary raw materials'. It encapsulates the circular economy's goal to enhance material recycling, reduce waste, and curb the reliance on primary raw material extraction.

{\large\textbf{Indicator Description:}}  \\ \indent
This indicator assesses the proportion of recycled material re-entering the economy against the overall material consumption, serving as a benchmark for the `circularity rate'. It signifies the efficiency of resource use by contrasting the circular use of materials against the aggregate domestic material consumption (DMC), adjusted for waste trade.

{\large\textbf{Unit:}}  \\ \indent
The indicator is presented as a percentage, depicting the share of recycled material in total material usage, reflecting the level at which secondary materials replace primary resources.

{\large\textbf{Source Data:}}  \\ \indent
Data is sourced from the European Statistical System (ESS) and Eurostat, employing a trio of statistical resources: waste treatment statistics, material flow accounts, and international trade data.

\subsubsubsection{CEI\_WM010: Recycling rate of all waste excluding major mineral waste}
\textbf{Indicator metadata:}   \href{https://ec.europa.eu/eurostat/cache/metadata/en/cei_wm010_esmsip2.htm}{\faExternalLink}

{\large\textbf{Context:}}  \\ \indent
This indicator is pivotal for measuring advancements in `waste management`. It gauges the efficiency of resource use by monitoring the volume of materials recycled and reincorporated into the economy, thus encapsulating the essence of material conservation and loss reduction.

{\large\textbf{Indicator Description:}}  \\ \indent
The recycling rate is formulated by the proportion of waste recycled versus the total waste treated, excluding significant mineral waste, rendered in percentage terms. It encompasses both hazardous and non-hazardous waste across all sectors, including household and secondary waste from waste treatment processes, thereby providing a comprehensive snapshot of the national recycling efforts.

{\large\textbf{Unit:}}  \\ \indent
Expressed in percentage

{\large\textbf{Source Data:}}  \\ \indent
Eurostat, under the aegis of the ESS, supplies this data. It incorporates waste treatment information aligned with the Waste Statistics Regulation, fine-tuned with international trade data, to accurately reflect the recycling of domestically produced waste.

\subsubsubsection{CEI\_WM011: Recycling rate of municipal waste}
\textbf{Indicator metadata:}   \href{https://ec.europa.eu/eurostat/cache/metadata/en/cei_wm011_esmsip2.htm}{\faExternalLink}

{\large\textbf{Context:}}  \\ \indent
As an integral part of the Circular Economy indicators, this measure serves as a barometer for the progression towards a more circular economy, with a focus on `waste management`. It assesses the re-utilisation of consumer waste in the economy, capturing the complexities inherent in the diverse composition of municipal waste.

{\large\textbf{Indicator Description:}}  \\ \indent
This indicator quantifies the proportion of municipal waste that is recycled, relative to the total amount of municipal waste produced, presented as a percentage. The breadth of municipal waste includes household refuse and similar commercial and public waste, representing a snapshot of the waste management quality from a consumer perspective.

\textit{"In order to comply with the objectives of this Directive, and move to a European circular economy with a high level of resource efficiency, Member States shall take the necessary measures designed to achieve the following targets: (a) by 2020, the preparing for re-use and the recycling of waste materials such as at least paper, metal, plastic and glass from households and possibly from other origins as far as these waste streams are similar to waste from households, shall be increased to a minimum of overall 50 \% by weight;"} --- Article 11.2 of the Waste Framework Directive.~\cite{eu2008wastedirective}

{\large\textbf{Unit:}}  \\ \indent
The metric of evaluation is a percentage

{\large\textbf{Source Data:}}  \\ \indent
Data source: European Statistical System (ESS)
Data provider: Statistical Office of the European Union (Eurostat) based on data reported by the countries: Municipal waste by waste operations \href{http://ec.europa.eu/eurostat/web/products-datasets/-/env_wasmun}{\faExternalLink} collected via a subset of the OECD/Eurostat Joint Questionnaire, section waste. Data are provided under a so-called gentlemen's agreement.

\subsubsubsection{CEI\_WM060: Recycling rate of waste of electrical and electronic equipment (WEEE) separately collected}
\textbf{Indicator metadata:}   \href{https://ec.europa.eu/eurostat/cache/metadata/en/cei_wm060_esmsip2.htm}{\faExternalLink}

{\large\textbf{Context:}}  \\ \indent
This indicator is a crucial component of the Circular Economy suite, offering insights into the progression towards enhanced sustainability in `waste management'. WEEE, or e-waste, is a rapidly expanding waste stream within the EU that encapsulates items like computers, TVs, refrigerators, and mobile phones. Given the valuable materials found in e-waste, improving recycling processes is of paramount importance.

{\large\textbf{Indicator Description:}}  \\ \indent
The indicator measures the efficiency of WEEE recycling by calculating the ratio of the weight of WEEE processed for recycling/re-use against the total weight of WEEE collected separately, in compliance with Article 11(2) of the WEEE Directive 2012/19/EU~\cite{eu2012weee, eu2012weeerecast}. The indicator's transition from `Recycling rate of e-waste' to its current form is to align more closely with the CE monitoring framework revisions.

The applicability of Directive 2012/19/EU is twofold:
\begin{itemize}
  \item Applicable up to the year 2018 for EEE classified under 10 product categories
        as outlined in Annex I of the Directive, with Annex II providing a
        corresponding indicative product list.
  \item Applicable from the year 2019 forward, where all EEE will be classified within
        6 product categories as delineated in Annex III.
\end{itemize}

{\large\textbf{Unit:}}  \\ \indent
The percentage serves as the unit of measure

{\large\textbf{Source Data:}}  \\ \indent \indent
Data procurement is executed by the ESS and supplied by Eurostat. The indicator's underlying data stems from:
\begin{itemize}
  \item For WEEE by waste operations:
        (env\_waselee)~\href{http://ec.europa.eu/eurostat/web/products-datasets/-/env_waselee}{\faExternalLink}.
  \item For WEEE by waste management operations - open scope, 6 product categories
        (from 2018 onwards):
        (env\_waseleeos)~\href{https://ec.europa.eu/eurostat/web/products-datasets/-/env_waseleeos}{\faExternalLink}.
\end{itemize}


\sectionEndlines
\clearpage
