\breaksection{Internal elements --- Technological Change}\label{sec:quantification_internal_rec}
\subsection{Introduction}

\boxgreen{\textbf{Technological Change}}{
        \textbf{Scenario elements}
        \begin{itemize}
          \item \textbf{Product technology (composition)} 
          \item \textbf{Recovery technology} 
          \item \textbf{Recovery system development}
        \end{itemize}
        \textbf{Waste model parameters include:}
        \begin{itemize}
          \item \textbf{Lifetimes:} \\ function of product technology
          \item \textbf{Composition:} \\ function of product composition, durability, design-for-repair, etc.
        \end{itemize}
        \textbf{Recovery model parameters include:}
        \begin{itemize}
          \item \textbf{Recovery processes:}\\  market penetration of recovery technologies
          \item \textbf{Transfer coefficients:} \\ function of recovery technologies
          \item \textbf{Recovery system size:} \\ BAU - set by trends in BAU, CIR \& REC - defined by model outcomes within constraints
        \end{itemize}
          }

        Product technology and recovery technology development will differ across scenarios and across product groups in each waste stream. These differences must be integrated into individual stock-flow models through interpretations of the scenarios and the conversion of these into estimations of the changes. 
        
        recognising the need for assumptions about technological advancements in products and recovery processes.  The document stresses the importance of addressing product technology development until 2050, as this will vary in importance across different waste streams.
          
        \subsection{Methodology}
          
        \begin{enumerate}
          \item \textbf{Scenario Analysis:}\newline
          Different future scenarios are explored to understand how technological development might unfold. This includes examining existing and new technologies, their properties over time, and how they are diffused into the market.
          
          \item \textbf{Waste Stream Focus:}\newline
          The approach varies in importance across different waste streams like EEE, ELV, BATT, CDW, and SLASH. For each stream, specific assumptions and approaches need to be tailored.
          
          \item \textbf{Data Collection:}\newline
          Gathering data from various sources, including patents, scientific publications, business reports, and expert opinions, to inform assumptions about technology development.
          
          \item \textbf{Modelling Approach:}\newline
          Utilizing dynamic Material Flow Analysis (dMFA) for modelling, covering a timeline from 2000 to 2050. This includes historical data analysis and future projections based on assumptions.
          
          \item \textbf{Technology Diffusion:}\newline
          Understanding the rate and manner in which technologies gain market share, depicted through S-curves. This involves considering factors like price, legislation, and organizational support.
          
          \item \textbf{Assumptions on Technology Development:}\newline
          Outlining specific assumptions at the level of each waste stream and aligning these assumptions across streams where possible.
          
          \item \textbf{Integration into SF Models:}\newline
          Incorporating the assumptions about product and recovery technology development into SF models to reflect the scenario-based variations.
          
          \item \textbf{Transparency and Communication:}\newline
          Ensuring that the chosen approaches and assumptions are clearly stated for transparency and ease of understanding.
      \end{enumerate}
      
      The methodology is designed to provide a comprehensive understanding of how future technology developments can impact various aspects of waste management and resource recovery in a circular economy context.
      

% table: the main technological scenario elements

% DOMAIN	ELEMENT	INTERNAL	EXTERNAL	OUTSIDE	BAU	REC	CIR	MODEL PARAMETERS AFFECTED
% ECO	Subsidies and taxation to promote recovery strategies	✔			I	III	I	recycling rates, recovery capacity, recovery impacts, collection
% POL	Targets and enforcement to promote recovery strategies	✔			I	III	I	recycling rates, recovery rates, capacity
% TECH	Product technology	✔			I	III	III	lifetimes, recovery rates, recovery impacts
% TECH	Recovery technology	✔			I	III	III	recovery rates, recovery capacity, recovery impacts
% TECH	Integration of SRM recovery system across Europe	✔			I	III	III	recycling rates, recovery rates, recovery capacity, recovery impacts

\clearpage

\subsection{Summary}

\boxreview{This summary will be compiled once the individual waste stream sections for each parameter are complete.}

\clearpage

% pull in the other subsections
\subsection{Future product and waste composition: \textit{Description}}

Future compositions, technologies and products will be assessed based on technology outlooks and stakeholder
interviews and may include sector-specific Delphi surveys. Information needs and availability for composition data
as well as the type of relevant recoverable embodied SRMs varies across the waste streams. Thus, specific data
collection strategies will be developed and used for each waste stream.

\boxblue{Task 2.2 and 2.3}{Following the scenarios from T2.1, the material compositions and future products for each sector will be determined based on the product and commodity demand and technology realisation (T2.2). This task will be coupled to the data collection in WP3 and WP4}

\subsubsection{Definition}


\boxparameter{Future product and waste composition}{Stable}{Strong shift toward design for design for recycling, remanufacturing, recovery}{Strong shift toward durability and design for re-X, especially re-use, repair}


\subsubsection{Method}

The general method for determining future product and waste composition will be based on the following steps:

\begin{itemize}
    \item \textbf{Step 1:} Collection of data on current product and waste composition (WP3)
    \item \textbf{Step 2:} Grouping of code lists into broader categories (WP3)
    \item \textbf{Step 3:} Identification of future products based on technology outlooks, literature review, and stakeholder interviews
    \item \textbf{Step 4:} Estimation of constraints for market entry, replacement, and market share of future products (WP2) in each scenario
    \item \textbf{Step 5:} Modelling of future product and waste composition based on the results of steps 1-4
\end{itemize}

% \subsubsection{Context}



% \subsubsection{International and European Trends}

% \subsubsection{Implementation in EU Law}


% \subsubsection{Development of a metric for XXX}

% \subsubsection{Benefits and risks}

% \subsubsubsection{Environmental Benefits and Risks}

% \subsubsubsection{Manufacturers' Perspective}


% \subsubsubsection{Broader Economic and Environmental Implications}


\subsubsection{Relevance of future product and waste composition to Critical Raw Materials in Waste Streams}

The integration of future product and waste composition has implications for the management of Critical Raw Materials
(CRMs) across various waste streams, such as BATT (waste batteries), ELV
(end-of-life vehicles), WEEE (waste electrical and electronic equipment), and
CDW (construction and demolition waste).

\subsubsubsection{Batteries (BATT)}
\begin{itemize}
    \item 
\end{itemize}

\subsubsubsection{End-of-Life Vehicles (ELV)}
\begin{itemize}
    \item 
\end{itemize}

\subsubsubsection{Waste Electrical and Electronic Equipment (WEEE)}
\begin{itemize}
    \item 
\end{itemize}

\subsubsubsection{Construction and Demolition Waste (CDW)}

\begin{itemize}
    \item
\end{itemize}

\sectionEndlines
\clearpage

\clearpage
\subsection{Future product and waste composition: \textit{Scenarios}}

\boxws{This section will be filled out with the details of exactly how this parameter is incorporated into your stock and flow models}

\subsubsection{Summary}

\boxreview{This summary will be compiled once the individual waste stream sections for each parameter are complete.}

\clearpage

\subsubsection[Scenario I: Business-as-usual]{\iconBAU Scenario I: Business-as-usual}


x


\wasteSubsubsubsecBATT

\wasteSubsubsubsecCDW

\wasteSubsubsubsecELV

\wasteSubsubsubsecMIN

\wasteSubsubsubsecSLASH

\wasteSubsubsubsecWEEE

\subsectionEndline
\clearpage

\subsubsection[Scenario II: Recovery]{\iconREC Scenario II: Recovery}

x

\wasteSubsubsubsecBATT

\wasteSubsubsubsecCDW

\wasteSubsubsubsecELV

\wasteSubsubsubsecMIN

\wasteSubsubsubsecSLASH

\wasteSubsubsubsecWEEE

\subsectionEndline
\clearpage


\subsubsection[Scenario III: Circularity]{\iconCIR Scenario III: Circularity}

x

\wasteSubsubsubsecBATT

\wasteSubsubsubsecCDW

\wasteSubsubsubsecELV

\wasteSubsubsubsecMIN

\wasteSubsubsubsecSLASH

\wasteSubsubsubsecWEEE

\sectionEndlines
\clearpage

\subsection{Future recovery technology: \textit{Description}}

\subsubsection{Definition}
This task will review current and emerging technologies used in the various sectors for product manufacturing and
end-of-life handling, with a special emphasis on material production, use, and recycling. Together with the storylines
developed in Task 2.1, it will adapt the market share of these technologies for each sector to determine the future
development of each sector.

\boxparameter{Future recovery technology}{Stable}{Strong increase}{Increase}

\subsubsection{Context}

Resource efficiency hinges on the effective integration and optimisation of product lifecycles, end-of-life (EOL) processing, the quality of recycled materials, recycling practices, and metallurgical technologies. The extent to which materials are diverted from landfills---owing to their complex make-up precluding economic recovery---is a key measure of this efficiency~\cite{reuter2012recyclinglimits}. Landfills, often the result of creating recyclates lacking economic viability, signify lapses in material management. While the second law of thermodynamics sets recyclability boundaries, these inefficiencies are frequently the consequence of preventable errors such as substandard product design, inefficient collection systems, and lack of process refinement.

To perpetuate resource availability and facilitate the flow of materials into sustainable products, it is imperative to establish well-conceived systems that reclaim these resources from EOL items and repurpose them. Grasping the influence of product design and the efficacy of recycling systems on closing material loops necessitates holistic methodologies that align with fundamental tenets, as delineated in this article.

To shape policy and engineer resource supply and recycling infrastructures, one requires an in-depth comprehension of recycling and high-temperature processing technologies, alongside insights into product design impacts and potential shifts in products and consumer behaviours. The formulation of a resilient system design is critical to amplifying resource efficiency---for instance, by minimising reliance on landfills---while also guaranteeing a consistent supply of metals for products within the renewable energy domain and other sectors pivotal to sustainability.

Resource efficiency is ultimately gauged by the proficiency in interlinking products, EOL processing, quality of recyclates, recycling, and metallurgical technology, and thereby determining how much material ends up in landfill due to its complex composition that negates economic value. Instances of poor material stewardship leading to landfills are attributed to the inability to generate economically feasible recyclates. While the second law of thermodynamics dictates the limitations of recyclability, such shortcomings are also attributable to avoidable blunders such as inadequate product design, collection systems, and process optimisation.



\subsubsection{Method}

The general method for determining future recovery technology will be based on the following steps:

\begin{itemize}
    \item \textbf{Step 1:} Collection of data on current recovery technology (WP4)
    \item \textbf{Step 2:} Identification of future recovery technology based on technology outlooks, literature review, and stakeholder interviews
    \item \textbf{Step 3:} Estimation of constraints for process data, market entry, replacement, and market share of future recovery technology in each scenario
    \item \textbf{Step 4:} Modelling of future recovery technology based on the results of steps 1-4
\end{itemize}

% Our research and development provides a theoretical basis for understanding the minimization of waste creation and hence the environmental burden of product and metal usage. It underpins resource efficiency with a theoretical basis, which is an important tool to help maintain and safeguard resources used in manufactured products, including scarce critical elements.

% 

% \subsubsection{International and European Trends}

% \subsubsection{Implementation in EU Law}


% \subsubsection{Development of a metric for XXX}

% \subsubsection{Benefits and risks}

% \subsubsubsection{Environmental Benefits and Risks}

% \subsubsubsection{Manufacturers' Perspective}


% \subsubsubsection{Broader Economic and Environmental Implications}


\subsubsection{Relevance of future recovery technology to Critical Raw Materials in Waste Streams}

The integration of the future recovery technology has implications for the management of Critical Raw Materials
(CRMs) across various waste streams, such as BATT (waste batteries), ELV
(end-of-life vehicles), WEEE (waste electrical and electronic equipment), and
CDW (construction and demolition waste).

\wasteSubsubsecBATT
\begin{itemize}
    \item
\end{itemize}

\wasteSubsubsecELV
\begin{itemize}
    \item
\end{itemize}

\wasteSubsubsecWEEE
\begin{itemize}
    \item
\end{itemize}

\wasteSubsubsecCDW
\begin{itemize}
    \item
\end{itemize}

\wasteSubsubsecMIN
\begin{itemize}
    \item
\end{itemize}

\wasteSubsubsecSLASH
\begin{itemize}
    \item
\end{itemize}
\sectionEndlines
\clearpage

\clearpage
\subsection{Future recovery technology: \textit{Scenarios}}

\boxws{This section will be filled out with the details of exactly how this parameter is incorporated into your stock and flow models}

\subsubsection{Summary}

\boxreview{This summary will be compiled once the individual waste stream sections for each parameter are complete.}

\clearpage

\subsubsection[Scenario I: Business-as-usual]{\iconBAU Scenario I: Business-as-usual}

x

\wasteSubsubsubsecBATT

\wasteSubsubsubsecCDW

\wasteSubsubsubsecELV

\wasteSubsubsubsecMIN

\wasteSubsubsubsecSLASH

\wasteSubsubsubsecWEEE

\subsectionEndline
\clearpage

\subsubsection[Scenario II: Recovery]{\iconREC Scenario II: Recovery}

x

\wasteSubsubsubsecBATT

\wasteSubsubsubsecCDW

\wasteSubsubsubsecELV

\wasteSubsubsubsecMIN

\wasteSubsubsubsecSLASH

\wasteSubsubsubsecWEEE

\subsectionEndline
\clearpage


\subsubsection[Scenario III: Circularity]{\iconCIR Scenario III: Circularity}

x

\wasteSubsubsubsecBATT

\wasteSubsubsubsecCDW

\wasteSubsubsubsecELV

\wasteSubsubsubsecMIN

\wasteSubsubsubsecSLASH

\wasteSubsubsubsecWEEE

\sectionEndlines
\clearpage
\clearpage
% \subsection{Future recovery system: \textit{Description}}

\subsubsection{Definition}



\boxparameter{Future recovery system}{Stable}{Strong increase}{Increase}


\subsubsection{Context}


\subsubsection{International and European Trends}

\subsubsection{Implementation in EU Law}


\subsubsection{Development of a metric for XXX}

\subsubsection{Benefits and risks}

\subsubsubsection{Environmental Benefits and Risks}

\subsubsubsection{Manufacturers' Perspective}


\subsubsubsection{Broader Economic and Environmental Implications}


\subsubsection{Relevance of XXX to Critical Raw Materials in Waste Streams}

The integration of the XXX has implications for the management of Critical Raw Materials
(CRMs) across various waste streams, such as BATT (waste batteries), ELV
(end-of-life vehicles), WEEE (waste electrical and electronic equipment), and
CDW (construction and demolition waste).

\wasteSubsubsecBATT
\begin{itemize}
    \item
\end{itemize}

\wasteSubsubsecELV
\begin{itemize}
    \item
\end{itemize}

\wasteSubsubsecWEEE
\begin{itemize}
    \item
\end{itemize}

\wasteSubsubsecCDW
\begin{itemize}
    \item
\end{itemize}

\wasteSubsubsecMIN
\begin{itemize}
    \item
\end{itemize}

\wasteSubsubsecSLASH
\begin{itemize}
    \item
\end{itemize}
\sectionEndlines
\clearpage

% \clearpage
% \subsection{Future recovery system: \textit{Scenarios}}

\boxws{This section will be filled out with the details of exactly how this parameter is incorporated into your stock and flow models}

\subsubsection{Summary}

\boxreview{This summary will be compiled once the individual waste stream sections for each parameter are complete.}

\clearpage

\subsubsection[Scenario I: Business-as-usual]{\iconBAU Scenario I: Business-as-usual}


x


\wasteSubsubsubsecBATT

\wasteSubsubsubsecCDW

\wasteSubsubsubsecELV

\wasteSubsubsubsecMIN

\wasteSubsubsubsecSLASH

\wasteSubsubsubsecWEEE

\subsectionEndline
\clearpage

\subsubsection[Scenario II: Recovery]{\iconREC Scenario II: Recovery}

x

\wasteSubsubsubsecBATT

\wasteSubsubsubsecCDW

\wasteSubsubsubsecELV

\wasteSubsubsubsecMIN

\wasteSubsubsubsecSLASH

\wasteSubsubsubsecWEEE

\subsectionEndline
\clearpage


\subsubsection[Scenario III: Circularity]{\iconCIR Scenario III: Circularity}
x
\wasteSubsubsubsecBATT

\wasteSubsubsubsecCDW

\wasteSubsubsubsecELV

\wasteSubsubsubsecMIN

\wasteSubsubsubsecSLASH

\wasteSubsubsubsecWEEE

\sectionEndlines
\clearpage

% template for including new subsections
% \input{133x-quantification_rec_Y_description}
% \input{133x-quantification_rec_Y_scenarios}

\subsubsection{Conclusion}


\boxreview{This conclusion will be compiled once the individual waste stream sections for each parameter are complete.}


\sectionEndlines
\clearpage
