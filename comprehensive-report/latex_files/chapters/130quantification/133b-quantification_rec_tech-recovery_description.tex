\subsection{Future recovery technology: \textit{Description}}

\subsubsection{Definition}
This task will review current and emerging technologies used in the various sectors for product manufacturing and
end-of-life handling, with a special emphasis on material production, use, and recycling. Together with the storylines
developed in Task 2.1, it will adapt the market share of these technologies for each sector to determine the future
development of each sector.

\boxparameter{Future recovery technology}{Stable}{Strong increase}{Increase}

\subsubsection{Context}

Resource efficiency hinges on the effective integration and optimisation of product lifecycles, end-of-life (EOL) processing, the quality of recycled materials, recycling practices, and metallurgical technologies. The extent to which materials are diverted from landfills---owing to their complex make-up precluding economic recovery---is a key measure of this efficiency~\cite{reuter2012recyclinglimits}. Landfills, often the result of creating recyclates lacking economic viability, signify lapses in material management. While the second law of thermodynamics sets recyclability boundaries, these inefficiencies are frequently the consequence of preventable errors such as substandard product design, inefficient collection systems, and lack of process refinement.

To perpetuate resource availability and facilitate the flow of materials into sustainable products, it is imperative to establish well-conceived systems that reclaim these resources from EOL items and repurpose them. Grasping the influence of product design and the efficacy of recycling systems on closing material loops necessitates holistic methodologies that align with fundamental tenets, as delineated in this article.

To shape policy and engineer resource supply and recycling infrastructures, one requires an in-depth comprehension of recycling and high-temperature processing technologies, alongside insights into product design impacts and potential shifts in products and consumer behaviours. The formulation of a resilient system design is critical to amplifying resource efficiency---for instance, by minimising reliance on landfills---while also guaranteeing a consistent supply of metals for products within the renewable energy domain and other sectors pivotal to sustainability.

Resource efficiency is ultimately gauged by the proficiency in interlinking products, EOL processing, quality of recyclates, recycling, and metallurgical technology, and thereby determining how much material ends up in landfill due to its complex composition that negates economic value. Instances of poor material stewardship leading to landfills are attributed to the inability to generate economically feasible recyclates. While the second law of thermodynamics dictates the limitations of recyclability, such shortcomings are also attributable to avoidable blunders such as inadequate product design, collection systems, and process optimisation.



\subsubsection{Method}

The general method for determining future recovery technology will be based on the following steps:

\begin{itemize}
    \item \textbf{Step 1:} Collection of data on current recovery technology (WP4)
    \item \textbf{Step 2:} Identification of future recovery technology based on technology outlooks, literature review, and stakeholder interviews
    \item \textbf{Step 3:} Estimation of constraints for process data, market entry, replacement, and market share of future recovery technology in each scenario
    \item \textbf{Step 4:} Modelling of future recovery technology based on the results of steps 1-4
\end{itemize}

% Our research and development provides a theoretical basis for understanding the minimization of waste creation and hence the environmental burden of product and metal usage. It underpins resource efficiency with a theoretical basis, which is an important tool to help maintain and safeguard resources used in manufactured products, including scarce critical elements.

% 

% \subsubsection{International and European Trends}

% \subsubsection{Implementation in EU Law}


% \subsubsection{Development of a metric for XXX}

% \subsubsection{Benefits and risks}

% \subsubsubsection{Environmental Benefits and Risks}

% \subsubsubsection{Manufacturers' Perspective}


% \subsubsubsection{Broader Economic and Environmental Implications}


\subsubsection{Relevance of future recovery technology to Critical Raw Materials in Waste Streams}

The integration of the future recovery technology has implications for the management of Critical Raw Materials
(CRMs) across various waste streams, such as BATT (waste batteries), ELV
(end-of-life vehicles), WEEE (waste electrical and electronic equipment), and
CDW (construction and demolition waste).

\wasteSubsubsecBATT
\begin{itemize}
    \item
\end{itemize}

\wasteSubsubsecELV
\begin{itemize}
    \item
\end{itemize}

\wasteSubsubsecWEEE
\begin{itemize}
    \item
\end{itemize}

\wasteSubsubsecCDW
\begin{itemize}
    \item
\end{itemize}

\wasteSubsubsecMIN
\begin{itemize}
    \item
\end{itemize}

\wasteSubsubsecSLASH
\begin{itemize}
    \item
\end{itemize}
\sectionEndlines
\clearpage
