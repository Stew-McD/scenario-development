\subsection{The Sharing Economy: \textit{Description}}\cite{dabbous2021sharing, pwc2016sharing, dggrow2018sharing, dggrow2018sharingenv, dabbous2021sharing, ps2share2017sharing, zhu2021sharing}

\subsubsection{Definition}
The sharing economy is a socio-economic system that emphasizes the
collaborative sharing of goods and services via community-based online
platforms. It represents a shift from traditional ownership, where assets were
exclusively leased, to a flexible model allowing for both personal use and
lease. This flexibility is a hallmark of the sharing economy, which has grown
significantly in response to advancements in technology, such as e-commerce and
mobile connectivity, coupled with a societal push for more sustainable living
and efficient resource use.


\boxparameter{The sharing economy}{Stable}{Stable}{Strong increase}


\subsubsection{Context}
As the concept of ownership transforms, particularly among the younger
generation, the sharing economy has increasingly taken root in the EU market.
This shift towards more communal and cost-effective ways of accessing goods and
services is supported by a new wave of consumer behavior, underpinned by
technological innovation and a pressing need to reduce environmental waste and
resource duplication.

While the sharing economy is broad and its definition fluid, it is often
associated with collaborative consumption, though the two can differ in motives
and mechanisms. Collaborative consumption may span consumer-to-consumer and
business-to-consumer interactions, whereas the sharing economy typically
operates within the consumer-to-consumer sphere. The sharing economy is thereby
defined as an innovative marketplace where entities engage in the distribution
and utilization of products and resources, with scalability achieved through
technological means.

This socio-economic model has not only disrupted traditional business sectors
but has also brought new value to the global economy, with rapid and profound
market penetration. Financial forecasts have been bullish, with revenue for
sharing platform providers expected to increase from USD 18.6 billion in 2017
to an estimated USD 40.2 billion in 2022. Moreover, the overall value of the
global sharing economy is projected to expand significantly, from USD 14
billion in 2014 to USD 335 billion in 2025, reflecting an unprecedented growth
trajectory over a mere twelve years\cite{zhu2021sharing}. Such growth reflects
the substantial economic potential and transformative power of the sharing
economy in contemporary markets.

\subsubsection{Scope within the EU Economy}
The sharing economy has made a significant economic contribution to the EU,
with an estimated €26.5 billion added to the GDP in 2016~\cite{pwc2016sharing}.
This figure is expected to grow, indicating the sharing economy's increasing
importance within the EU's economic structure.

\subsubsection{Environmental Prospects}~\cite{dabbous2021sharing, pwc2016sharing, dggrow2018sharingenv}

See~\cite{zhu2021sharing} Table 2 for a summary of the studies on the
environmental impacts of the sharing economy.

The sharing economy has the potential to reshape consumption behaviors and
reduce environmental impacts by promoting the sustainable use of resources.
This economic paradigm encourages the efficient employment of underutilised
goods, which can lead to a decrease in the need for new products, thus
conserving resources and mitigating greenhouse gas emissions. It fosters a
lifestyle that lessens the adverse environmental effects of consumption while
improving quality of life.

Central to the sharing economy is the promotion of moderate consumption
patterns. This approach aims to reduce the excessive purchasing habits of
certain populations to alleviate ongoing environmental harm. The sharing
economy's alignment with green consumption practices encompasses waste
reduction, energy conservation, and the adoption of sustainable resources, all
while managing and moderating excessive consumption.

The impact on the fast fashion industry serves as a pertinent example, with the
sector's frequent turnover to keep pace with changing trends leading to
significant textile waste. Collaborative consumption through the sharing
economy can mitigate this waste by encouraging the reuse and extension of
clothing's service life. Clothing libraries are an example of how the sharing
economy can provide environmental benefits by prolonging the usable life of
garments.

Eco-efficiency is enhanced when environmental resources are utilised more
effectively, leading to an increased use of products with minimal environmental
burden. This is exemplified in collaborative fashion consumption, which could
reduce the prevalent overconsumption in the fashion industry. By facilitating
the exchange of underused clothing, the sharing economy can increase the
lifecycle of garments and encourage the production of more durable products.

Beyond the realm of fashion, car sharing and shared accommodation are other
aspects of the sharing economy with notable environmental benefits. Car sharing
can significantly reduce the number of vehicles needed, thereby lowering
exhaust emissions. Similarly, shared accommodations have been associated with
significantly lower carbon dioxide emissions compared to conventional hotel
stays.

However, the question of whether the sharing economy indeed delivers
environmental benefits remains contested.~\cite{zhu2021sharing,
  geissinger2019sharing} Detractors highlight the potential for an increase in
environmental burdens, particularly if the heightened usability of shared goods
escalates greenhouse gas emissions. The environmental and socio-economic
impacts engendered by the collaborative economy are intricate and highly
variable across different business models. Generally speaking, collaborative
consumption models that optimise the use of existing assets tend to exhibit a
lower environmental footprint compared to their traditional counterparts.
Nevertheless, there is a risk that the financial savings afforded by
collaborative consumption could spur additional spending and consumption, which
might negate the direct environmental savings. Despite such reservations, the
prevailing view is that the sharing economy, by transforming consumption from
ownership to communal use, can yield considerable environmental advantages.

\subsubsection{Implications for Waste Streams}

The adoption of sharing economy principles can influence various waste streams,
including:

\begin{itemize}
  \item \textbf{BATT (Waste Batteries):} As devices are shared and utilized more efficiently, the frequency of battery disposal could decline, mitigating the waste battery stream.
  \item \textbf{CDW (Construction and Demolition Waste):} The sharing of construction equipment and machinery could potentially slow down the turnover rate of these items, reducing associated waste.
  \item \textbf{WEEE (Waste Electrical and Electronic Equipment):} Sharing electronic devices extends their lifecycle and reduces the rate at which they are discarded, thereby impacting electronic waste volumes.
  \item \textbf{ELV (End-of-Life Vehicles):} A shift towards car-sharing services could reduce the demand for manufacturing new vehicles, potentially leading to a downturn in the generation of automotive waste.
\end{itemize}

The trajectory of the sharing economy indicates a shift towards collective
usage patterns. Its continuing evolution could play a critical role in the
future of critical raw material recovery systems by affecting demand and the
lifecycle of products, which, in turn, influences waste stream outputs. The
broader implications for the raw materials sector are significant, suggesting a
possible recalibration of recovery strategies for critical raw materials in
light of emerging consumption patterns.

\subsubsection{Challenges in Measuring Sharing Economy Growth}

Identifying a universal metric for the growth of the sharing economy is
challenging due to its diverse and dynamic nature. Current measures, such as
STOXX Global sharing economy indices, Solactive Sharing Economy Index and the
INDXX US Sharing Economy Index, largely revolve around the market sizes of
prominent sharing economy companies, such as Uber and Airbnb. These indices,
while useful, predominantly reflect the scalability of these businesses rather
than the sharing economy's broader impacts on production and waste reduction.

In the absence of a standardised metric, the assessment of the sharing
economy's expansion is often best approached through product-specific data.
This involves examining the adoption rates and usage trends of sharing services
at the product level to infer growth patterns. Such a detailed, product-centric
analysis allows for a closer inspection of the sharing economy's implications
on resource utilisation and waste generation, offering insights that aggregate
economic data may overlook.

\sectionEndlines
\clearpage