\subsection{Refuse, Reduce, Reuse: \textit{Description}}

Refuse, Reduce, and Reuse are the first three short loops (R0-2) in the circular economy scheme. They exist close to the 
consumer and can be linked to commercial or non-commercial actors engaged in extending the product's lifespan~\cite{vermeulen2019rex}.

\boxparameter{Refuse, reduce, reuse}{Stable}{Stable}{Strong increase}


These strategies are pivotal in the circular economy, effectively reducing the environmental impact of products or services.

\begin{itemize}
    \item \textbf{R0 --- Refuse:} Choosing not to purchase products or services that are unnecessary or unsustainable. 
    \item \textbf{R1 --- Reduce:} Decreasing the quantity of products or services used or needed.
    \item \textbf{R2 --- Reuse:} Utilising products or services again for the same or a different purpose.
\end{itemize}

These strategies will be incorporated into FutuRaM's modelling framework through:

\begin{itemize}
    \item Waste volume reduction from:
    \begin{itemize}
        \item Reduction in overall demand, that is, the put on market (POM) of a product or service.
        \item Reduction in the amount of material used in a product or service (efficiency).
        \item Extension of the lifetime of products from reuse.
    \end{itemize}
    \item Changes in the composition of waste, as some products are more amenable to being refused or reused than others. 
\end{itemize}

\subsubsection{Definitions}

\subsubsubsection{Refuse}
Refuse encompasses consumer and producer decisions aimed at minimising waste creation and reducing environmental impact. For consumers, it involves choosing not to purchase products that are not environmentally friendly and reducing overall consumption. In the production context, it signifies the deliberate avoidance of certain materials or processes to enhance circularity, such as eschewing hazardous substances or designing to minimise waste. Refuse, as a concept, prioritises waste prevention at the source and is integral to fostering a more circular economy~\cite{reike2018rex}.

\subsubsubsection{Reduce}
Reduce refers to strategies aimed at minimising the use of natural resources, including energy, raw materials, and thereby reducing waste generation. This concept is multifaceted:

\begin{itemize}
    \item For consumers, it involves using products less frequently, caring for and repairing products to extend their life, and participating in the sharing economy.
    \item For producers, it focuses on using less material per production unit, often referred to as 'dematerialisation', and incorporating these principles early in the Concept and Design Life Cycle.
\end{itemize}

Reduction is also linked to the notion of Reuse, as decreasing the quantity of products (like cars) can incentivise their reuse. Policy measures to enforce reduction, such as banning single-use plastics or imposing environmental taxes, are essential for effective implementation~\cite{maitreekern2019rex}.

\subsubsubsection{Reuse}
Reuse is about extending the life of products in their original form for as long as possible, thus conserving resources and energy. It involves maintaining and repairing products to keep them in use and developing business models that support these practices. Examples include:

\begin{itemize}
    \item Reusable packaging initiatives in various industries.
    \item Encouraging the reuse of items like clothing, furniture, and electronics.
    \item Deposit-refund systems that incentivise product return and reuse.
\end{itemize}

Reuse strategies are vital for reducing the consumption of new products and avoiding the dichotomy of 'new for the rich, reused for the poor', promoting equitable and sustainable consumption patterns~\cite{morseletto2020cetargets}.



% \subsubsection{International and European Trends}

% \subsubsection{Implementation in EU Law}


% \subsubsection{Development of a metric for XXX}

% \subsubsection{Benefits and risks}

% \subsubsubsection{Environmental Benefits and Risks}

% \subsubsubsection{Manufacturers' Perspective}


% \subsubsubsection{Broader Economic and Environmental Implications}


% \subsubsection{Relevance of refuse, reduce and reuse to Critical Raw Materials in Waste Streams}

% The integration of the the re-X stragegies of refuse, reduce and reuse has implications for the management of Critical Raw Materials
% (CRMs) across various waste streams, such as BATT (waste batteries), ELV
% (end-of-life vehicles), WEEE (waste electrical and electronic equipment), and
% CDW (construction and demolition waste).

% \wasteSubsubsecBATT
% \begin{itemize}
%     \item
% \end{itemize}

% \wasteSubsubsecELV
% \begin{itemize}
%     \item
% \end{itemize}

% \wasteSubsubsecWEEE
% \begin{itemize}
%     \item
% \end{itemize}

% \wasteSubsubsecCDW
% \begin{itemize}
%     \item
% \end{itemize}

% \wasteSubsubsecMIN
% \begin{itemize}
%     \item
% \end{itemize}

% \wasteSubsubsecSLASH
% \begin{itemize}
%     \item
% \end{itemize}
\sectionEndlines
\clearpage
