\breaksection{Overview of the scenario storylines}

\subsection*{\parbox{\linewidth}{\centering\Large Scenario I: Business-as-usual (BAU)}}

\vspace{-1em}
\iconBAUbig
\vspace{-1em}
See \autoref{sec:bau} for the full scenario description and waste-stream-specific scenario impact narratives.

This scenario envisions the future based on the current situation, extending to 2050 with very little deviation from present consumption patterns and without substantial development of the secondary raw material (SRM) recovery system. While there may be advances in some areas such as resource efficiency, recovery technology, and the energy transition, substantial modifications remain hindered by economic, social, and political constraints. The extraction of primary raw materials continues to be the predominant source utilised to satisfy the EU's growing SRM demand.


\textbf{In the Business as usual (linear economy) scenario, the following are key characteristics:}
\begin{itemize}
\item A forecasting model is used to predict the future based on the current situation and the development of existing trends.
\item EU targets including those for eco-design, recycling and recovery are not met, and the current linear model largely persists.
\item Material demand remains coupled with economic growth, perpetuating a trend of increasing consumption. 
\item Primary mining and extraction persist as the leading sources of raw materials, underlining the dependency on traditional extraction methods.
\item Recycling and recovery rates continue to lag, leading to increased production of SRM-containing waste that signals missed opportunities for resource reuse.
\item The EU's dependency on imports of SRMs escalates, heightening the risk of supply disruptions~\cite{iea2023crm}. 
\item Investment in new SRM recovery technologies remains minimal, stifling innovation and advancements in this field.
\item The industrial focus remains on cost-effective material production and use, disregarding the long-term sustainability aspect.
\item Material scarcity and price fluctuations pose potential risks to the EU industry, highlighting the vulnerability of this business model~\cite{mancini2013supplysecurity}.
\item Without any significant updates to environmental regulations, the negative impacts on ecosystems and biodiversity intensify.
\item Mining activity in the EU remains limited and concentrated in only a few member states. Current exploration projects (e.g., for Lithium in PT, FR, UK and rare earths in SE) are not realised.
\item The transitions to renewable energy and e-mobility continue at their current pace.
\end{itemize}

\subsectionEndline
\clearpage

\subsection*{\parbox{\linewidth}{\centering\Large Scenario II: Recovery (BAU)}}

\iconRECbig

See \autoref{sec:rec} for the full scenario description and the waste-stream-specific scenario impact narratives.

In the recovery scenario, the central emphasis is on harnessing sophisticated technologies to salvage SRMs from waste streams at the end of their lifecycle. While there are noticeable strides towards the incorporation of `circular design' principles and re-X strategies (which focus on reducing, reusing, recycling, repairing, and refurbishing), material demand increases similarly to the BAU scenario. This is, however, mitigated to some extent by the implementation of a comprehensive material recovery system.



\textbf{Key features of this technology-promoted recovery scenario include:}
\begin{itemize}
\item This scenario uses a combination of forecasting and backcasting methods to envision the future.
\item The backcasting method is used for scenario factors that are covered by governmental targets, starting with the desired outcome and working backwards to the present.
\item The forecasting method is used for scenario factors that are not covered by governmental targets, starting with the current situation and extending to the future.
\item EU targets for recycling and recovery are met, due to the EU's waste management system becoming more expansive, efficient and effective.
\item Technological innovation drives increased recovery rates of SRMs, enabling the more efficient use of waste.
\item Digitalisation and automation are more extensively used in recycling processes, leading to enhanced productivity and efficiency.
\item Business models like leasing and take-back schemes emerge, altering traditional consumption patterns (here, the focus is on take-back for recycling).
\item Ecodesign mandates are implemented, again, here, with a focus on end-of-life recovery.
\item There is greater exploration and exploitation of alternative sources such as urban mining, waste streams, and tailings, presenting novel opportunities for resource acquisition.
\item New waste regulations and guidelines for SRM recovery are implemented, enforcing better management and extraction of SRMs.
\item Investment in research and development for SRM recovery technologies experiences an upswing, promoting continuous innovation in this field.
\item Closer collaboration and information sharing between industry and government institutions (e.g., waste tracking and digital product passports) streamline processes and expedite decision-making.
\item New jobs are created in the recycling and recovery sector, offering economic benefits and improving overall employment rates.
\item SRM production and use become more efficient and cost-effective, fostering economic sustainability.
\end{itemize}

\subsectionEndline
\clearpage

\subsection*{\parbox{\linewidth}{\centering\Large Scenario III: Circularity}}

\iconCIRbig

See \autoref{sec:cir} for the full scenario description and the waste-stream-specific scenario impact narratives.

In this scenario, we move in the direction of the maximum achievable state of material efficiency as government policy, private innovation and social changes are rapidly driving the transition toward a circular economy. The emphasis here rests heavily on re-X strategies that are implemented in the design phase of products (e.g., repairability and re-manufacturability) and that are actualised by changes in consumer behaviour (e.g reduction, refusal, engagement in the `sharing economy' and curtailment of the `throw-away' mindset). 

Further, being enabled by the widespread adoption of `circular design' principles and improvements in information transparency (e.g., waste tracking and digital product passports) the system for the treatment of post-consumer waste can divert a significant amount of their inflows (to, for example, re-use and re-manufacture) with the residual fraction being readily segregated into purer, more efficiently recoverable, material streams. 

This scenario envisions a future where government policies are in synergy with private sector innovation and societal changes, driving a wholesale transition towards a circular economy. Unlike the recovery scenario, where the focus is on the end-of-life recovery of materials, this scenario emphasises minimising waste at all stages, starting from the design phase itself.

\textbf{The circular economy scenario is characterised by the following:}
\begin{itemize}
\item This scenario uses a combination of forecasting and backcasting methods to envision the future.
\item The backcasting method is used for scenario factors that are covered by governmental targets, starting with the desired outcome and working backwards to the present.
\item The forecasting method is used for scenario factors that are not covered by governmental targets, starting with the current situation and extending to the future.
\item EU targets for recycling and recovery are met, as are those for circularity, due to advances in waste management, ecodesign and re-X strategies.
\item A circular economy is implemented, prioritising waste reduction, resource efficiency, and a shift from the `take-make-dispose' model.
\item A notable increase in SRM recycling and recovery rates, indicating an efficient use of resources.
\item A larger emphasis on designing products for reuse and recycling, making waste a valuable resource rather than a problem.
\item More extensive use of renewable energy and clean technologies in SRM production and use, supporting a low-carbon economy.
\item Collaboration between stakeholders --- including industry, government, and consumers --- improves, enhancing the implementation of circular practices.
\item New business models like leasing and take-back schemes emerge, altering traditional consumption patterns~\cite{geissdorfer2020circbusinessmodels}.
\item Digitalisation and data use are heightened to improve efficiency and traceability, aiding in effective resource management.
\item Investment in research and development for circular economy technologies increases, driving innovation and adoption.
\item Awareness and education around sustainable consumption and production practices are amplified, leading to behavioural changes in society.
\item Reliance on imports decreases, suggesting greater self-sufficiency and sustainability.
\item The creation of new jobs within the recycling, recovery and re-X sectors boosts the economy and alleviates social inequality.
\item Stricter waste regulations and product design guidelines are introduced, accelerating the transition towards circularity.
\end{itemize}

\sectionEndlines
\clearpage