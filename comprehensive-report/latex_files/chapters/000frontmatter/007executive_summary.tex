\breaksection{Executive Summary}\label{sec:executivesummary}

This report presents the first phase of the scenario development process — the storyline narrative phase. Three distinct future scenarios have been drafted up to the year 2050: Business as Usual, Recovery, and Circularity. These scenarios are designed to be internally consistent and provide an overview of the potential future landscape of waste management and SRM recovery within the EU.

The scenario development process employs a methodology that integrates both forecasting and backcasting techniques to build a comprehensive, future-facing knowledge base that can aid fact-based decision-making~\cite{dreborg1996essence, boerjeson2005scenariosreport, boerjeson2005scenariosarticle, amer2013, sardesai2021, lloyd2014objectivity, ardente2014}. 

In the next phases of scenario development, future product composition and recovery technology will be assessed, scenario elements will be quantified, and all data will be integrated with the quantitative models for waste generation and SRM recovery.

The FutuRaM project aims to offer a nuanced understanding of the potential future waste management and resource recovery landscape within the EU. This approach provides insights into key drivers, uncertainties, and the possible impacts of policy and technological advancements. Additionally, by aligning SRM recovery efforts with the United Nations Framework Classification for Resources (UNFC)~\cite{unfc2023}, the project aims to facilitate the commercial exploitation of SRMs and CRMs by manufacturers, recyclers, and investors. With the comprehensive knowledge base that we are developing, FutuRaM aims to support informed decision-making by policymakers and government, as well as industry and community stakeholders.


\vspace{2em}
\begin{center}
    {\Large\textbf{\textsc{FutuRaM's Three Future Scenarios}}} \\
    \rule{.6\textwidth}{2pt}
\end{center}

\vspace{1em}
\begin{samepage}
    
    \iconBAUbig
    
    \subsection*{\parbox{\linewidth}{\centering\Large Scenario I: Business-as-usual (BAU)}}
    
    The BAU scenario extrapolates current trends into the future with limited change. Using forecasting techniques, it projects a potential future where there are minor advancements in resource efficiency, recovery technology, and the energy transition, but primary extraction of raw materials remains the dominant practice.
\end{samepage}
    
\begin{samepage}
    
    \iconRECbig
    
    \subsection*{\parbox{\linewidth}{\centering\Large Scenario II: Recovery (REC)}}
    
    The Recovery scenario imagines a future leveraging advanced technology to significantly enhance SRM recovery from waste streams. It outlines a future where the EU successfully meets its recycling and recovery targets through an effective waste management system and circular design principles~\cite{halleux2021batt,helander2023battelv}. This scenario sees an increased recovery rate of SRMs, extensive use of digitalisation and automation in recycling processes, and new or strengthened waste regulations in line with EU targets.
\end{samepage}

\begin{samepage}
    
    \iconCIRbig
    
\subsection*{\parbox{\linewidth}{\centering\Large Scenario III: Circularity (CIR)}}

The Circularity scenario captures the ideal of a fully realised circular economy, going beyond end-of-life recovery to minimise waste at every production and consumption stage. It predicts a future where the EU's targets for recycling, recovery, and circularity are met through extensive stakeholder collaboration, new business models, and increased use of renewable energy and circular economy technologies~\cite{eu2020circ,kirchherr2017circ, domenech2019transition}.


\end{samepage}
\sectionEndlines
